% This file is part of the Bart project.
% Copyright 2013 the Authors.

\documentclass[12pt,letterpaper,preprint]{aastex}

\newcommand{\Kepler}{\textsl{Kepler}}

\begin{document}

\title{Imaging the surfaces of stars with exoplanet transits}
\author{some combination of TB, DFM, DWH, others?}

\begin{abstract}
The shape of an exoplanet transit is set by the relative sizes of the planet and
star, the impact parameter (inclination) and the limb-darkening model, plus
additional effects from sunspots, oblateness, orbital eccentricity, and so on.
%
This makes it possible to infer a star's limb-darkening profile from
observations of planets transiting the star.
%
We perform this analysis for [star identifier], a [star type] hosting [number]
transiting planets.
%
We fit the limb-darkening profile with a flexible (step-function or zeroth-order
spline) model; we obtain a posterior PDF for the parameters of this model (step
heights), marginalizing out the other parameters of the system, including
exoplanet orbital parameters, star and planet radii, and inclination.
%
We find that the theoretically expected limb-darkening profile is strongly ruled
out.
%
The principal result of this study is that limb-darkening profiles \emph{can} be
learned from transits, and that there is great promise in a data-driven model
for stellar limb darkening.
\end{abstract}

\keywords{
foo
---
bar
}

\section{Introduction}

...Studies that have figured stuff out about stellar rotation.

...Studies that have figured stuff out about sunspots.

...All our knowledge about the properties of exoplanets is limited by our
knowledge about the properties of stars.

\section{Exoplanet transit inference}

\section{Limb-darkening profiles}

\section{Discussion}

\acknowledgements
Thanks to the \Kepler\ team.

\end{document}
